\title{PSE Entwurf-Muster}
\subtitle{Anwenderorientierte Nutzerschnittstelle für Luftqualitätsdaten Team 2}
\date{\today}
\author{Katharina Biernacka \and Maria Kraus \and Fabian Reinbold \and Daniel Schild}

%Commands
\newcommand{\softwarename}{VisAQ }
\newcommand{\longSoftwarename}{Visualizing Air Quality}
\newcommand{\class}[1]{\paragraph{#1}\mbox{}\\}

%Commands for descriptions of often used methods and parameters
\newcommand{\equalsDescription}[1]{Vergleicht zwei Elemente hinsichtlich der Gleichheit.
Wird ein Objekt eines anderen Typs als #1 übergeben, so wird false zurück gegeben.
Zwei Elemente vom Typ #1 gelten als gleich, wenn alle Klassenattribute gleich sind.
Die Methode wird von der Java-Klasse java.lang.Object deklariert und hier überschrieben.}
\newcommand{\equalsDescriptionWithDB}[1]{Vergleicht zwei Elemente hinsichtlich der Gleichheit.
Wird ein Objekt eines anderen Typs als #1 übergeben, so wird false zurück gegeben.
Zwei Elemente vom Typ #1 gelten als gleich, wenn alle Klassenattribute gleich sind. Also explizit dann wenn sie den selben Datensatz in der Datenbank repräsentieren.
Die Methode wird von der Java-Klasse java.lang.Object deklariert und hier überschrieben.}
\newcommand{\constructorDescription}[1]{Konstruktor für ein Objekt des Typs #1.}
\newcommand{\multiLinkDescription}[1]{Ein MultiNavigationLink der auf mehrere Objekte des Typs #1 zeigt.}
\newcommand{\singleLinkDescription}[1]{Ein SingleNavigationLink der auf ein Object des Typs #1 zeigt.}
\newcommand{\sensorthingsClassDescription}[2]{Die Klasse #1 stellt einen Datensatz aus der #2-Tabelle der \gls{SensorThings API}.}
\newcommand{\relativeDescription}{Die boolean relative sagt aus, ob der Link ein absoluter Link zur Datenbank ist, oder nur relativ zum in RestConstants gegebene Entry-Point liegt.}