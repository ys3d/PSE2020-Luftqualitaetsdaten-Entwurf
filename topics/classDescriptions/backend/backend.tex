\subsection{Backend}
%VisAQ
\class{VisAQ}
public class VisAQ
\\\\
\begin{minipage}{0.3\textwidth}
    \begin{figure}[H]
        {\centering\includegraphics[width=0.95\textwidth]{media/backend/classes/VisAQ.png}}
    \end{figure}
    \end{minipage} \hfill
\begin{minipage}{0.7\textwidth}
    Die Klasse VisAQ stellt den Einstiegspunkt ins Backend da.
\end{minipage}

Methoden:
\begin{itemize}
    \item \emph{public static void main(String[] args)} Die main-Methode des Programms.
    Über diese Methode wird das Programm gestartet. Ein Array von Strings kann dem Start als Parameter standardmäßig mitgegeben werden, hier werden diese Parameter jedoch nicht aktiv genutzt
\end{itemize}
\subsubsection{Model}
%abstract classes%%%%%%%%%%%%%%%%%%%%%%%%%%%%%%%%%%%%%%%%%%%%%%%%%%%%%%%%%%%%%%%%%%%
% Sensorthing<SensorthingT> 
\rule{\textwidth}{0.4pt}
\class{Sensorthing<SensorthingT>}
\begin{minipage}{0.4\textwidth}
    \begin{figure}[H]
        {\centering\includegraphics[width=0.95\textwidth]{media/backend/modell/classes/Sensorthing.png}}
    \end{figure}
    \end{minipage} \hfill
    \begin{minipage}{0.6\textwidth}
Die abstrakte Klasse Sensorthing stellt ein Objekt aus der \gls{SensorThings API} da.
Die Klasse vereint die gemeinsamen Eigenschaften der Klassen aus der Datenbank-\gls{API} in sich.
Um die Datentypen der Parameter möglichst exakt zu bestimmen werden \glspl{bounded quantification} verwendet.
\end{minipage}

Attribute:
\begin{itemize}
    \item \emph{public final String id} Jedes Objekt der Datenbank-\gls{API} ist mit einer eindeutigen ID versehen, die das Objekt identifiziert.
    Da sich die ID aus Buchstaben, Zeichen und Ziffern zusammensetzt wird sie hier durch einen String repräsentiert.
    \item \emph{public final SingleLocalLink<SensorthingT> selfLink} Jedes Objekt der \gls{SensorThings API} hält einen Verweis auf die Online-Instanz von sich selbst.
    So kann die Herkunft und die übereinstimmung des Datensatzes mit der Online-Version jederzeit überprüft werden.
\end{itemize}
Methoden: \begin{itemize}
    \item \emph{public Sensorthing(String id, String selfURL, boolean relative)} Diese Methode ist er einzige Konstruktor der Sensorthing-Klasse.
    Als Parameter erhält der Konstruktor neben dem Klassen-Attribut id auch eine URL und eine boolean relative. Beide Parameter werden zum initialisieren des SingleLocalLink benötigt.
\end{itemize}

%%Interfaces%%%%%%%%%%%%%%%%%%%%%%%%%%%%%%%%%%%%%%%%%%%%%%%%%%%%%%%%%%%%%%%%%%%
% SensorthingsProperties
\rule{\textwidth}{0.4pt}
\class{SensorthingsProperties}
\begin{minipage}{0.4\textwidth}
    \begin{figure}[H]
        {\centering\includegraphics[width=0.95\textwidth]{media/backend/modell/classes/SensorthingsProperties.png}}
    \end{figure}
    \end{minipage} \hfill
\begin{minipage}{0.6\textwidth}
Die Schnittstelle SensorthingsProperties zeigt an, dass ein Objekt der \gls{SensorThings API} Properties besitzt.
Properties sind in diesem Fall Daten vom Typ Object die durch einen String als eindeutigen Identifier gesucht werden können.
Es wird nicht näher spezifiziert welche Properties ein entsprechendes Object muss.
\end{minipage}

Methoden: \begin{itemize}
    \item \emph{public Object getPropertyByKey(String key)} Die Methode sucht in den properties des Objektes eine Eigenschaft mit dem gegebenen Key und gibt den Wert dieser zurück.
    Wenn keine passende Eigenschaft gefunden wurde wird null zurück gegeben.
    \item \emph{public boolean hasProperties(String key)} Die Methode überprüft ob eine Eigenschaft mit dem gegebene Key existiert.
    Falls eine solche Eigenschaft existiert wird true zurück gegeben, andernfalls false.
\end{itemize}

% SensorthingsTimeStamp
\rule{\textwidth}{0.4pt}
\class{SensorthingsTimeStamp}
\begin{minipage}{0.4\textwidth}
    \begin{figure}[H]
        {\centering\includegraphics[width=0.95\textwidth]{media/backend/modell/classes/SensorthingsTimeStamp.png}}
    \end{figure}
    \end{minipage} \hfill
\begin{minipage}{0.6\textwidth}
Die Schnittstelle SensorthingsTimeStamp zeigt an, dass ein Objekt der \gls{SensorThings API} einen Zeitstempel besitzt.
Ein Zeitstempel impliziert in diesem Fall, dass zwei solche Objekte anhand ihres Zeitstempels verglichen werden können.
\end{minipage}

Methoden: \begin{itemize}
    \item \emph{public boolean isOlder(SensorthingsTimeStamp other)} Die Methode prüft ob die Instanz älter als die mit other gegebene Instanz ist.
    In diesem Fall wird true zurückgegeben, ansonsten false.
    \item \emph{public boolean isNewer(SensorthingsTimeStamp other)} Die Methode prüft ob die Instanz jünger als die mit other gegebene Instanz ist.
    In diesem Fall wird true zurückgegeben, ansonsten false.
    \item \emph{public boolean isEqualOld(SensorthingsTimeStamp other)} Die Methode prüft ob die Instanz gleich alt wie die mit other gegebene Instanz ist.
    In diesem Fall wird true zurückgegeben, ansonsten false.
    \item \emph{public Instant getTimeStamp()} Die Methode gibt den Zeitstempel als java.time.Instant zurück.
\end{itemize}

%Classes%%%%%%%%%%%%%%%%%%%%%%%%%%%%%%%%%%%%%%%%%%%%%%%%%%%%%%%%%%%%%%%%%%%%%%%
%Datastream
%TODO: Fix class-table size
\rule{\textwidth}{0.4pt}
\class{Datastream extends Sensorthing<Datastream> implements SensorthingsProperties}
\begin{minipage}{0.4\textwidth}
    \begin{figure}[H]
        {\centering\includegraphics[width=0.95\textwidth]{media/backend/modell/classes/Datastream.png}}
    \end{figure}
    \end{minipage} \hfill
\begin{minipage}{0.6\textwidth}
    \sensorthingsClassDescription{Datastream}{Datastream}
    \url{http://developers.sensorup.com/docs/#datastreams_post}
\end{minipage}

Attribute:
\begin{itemize}
    \item \emph{public final String name} Dieser String repräsentiert den Namen des Datastreams in der Datenbank.
    \item \emph{public final String description} Dieser String stellt repräsentiert die Beschreibung des Datstreams in der Datenbank.
    \item \emph{public final UniteOfMeasurement unitOfMeasurement} Jedem Datastream ist in der Datenbank eine Maßeinheit zugeordnet. Diese maßeinheit wird hier als UniteOfMeasurement gekapselt gespeichert.
    \item \emph{public final String observationTypeLink} Der observationTypeLink gibt an um welchen Typ von Datastream es sich handelt.
    In der Regel wird hier ein Link auf eine Online-Erklärung des Datastream-Typs abgebildet.
    \item \emph{public final MultiNavigationLink<Observation> observationsLink} \multiLinkDescription{Observation}
    \item \emph{public final SingleNavigationLink<Sensor> sensorLink} \singleLinkDescription{Sensor}
    \item \emph{public final SingleNavigationLink<Thing> thingLink} \singleLinkDescription{Thing}
\end{itemize}
Methoden: \begin{itemize}
    \item \emph{public Datastream(String id, String selfUrl, boolean relative, String name, String description, Map<String, Object> properties, String observationTypeLink, SingleNavigationLink<Sensor> sensorLink, SingleNavigationLink<Thing> thingLink, MultiNavigationLink<Observation> observationsLink, UnitOfMeasurement unitOfMeasurement, SingleNavigationLink<ObservedProperty> observedPropertyLink)}
    \constructorDescription{Datastream}
    \item \emph{public boolean equals(Object obj)} \equalsDescription{FeatureOfInterest}
\end{itemize}

%FeatureOfInterest
%TODO: Fix class-table size
\rule{\textwidth}{0.4pt}
\class{FeatureOfInterest extends Sensorthing<FeatureOfInterest>}
\begin{minipage}{0.4\textwidth}
    \begin{figure}[H]
        {\centering\includegraphics[width=0.95\textwidth]{media/backend/modell/classes/FeatureOfInterest.png}}
    \end{figure}
    \end{minipage} \hfill
\begin{minipage}{0.6\textwidth}
    \sensorthingsClassDescription{FeatureOfInterest}{FeaturesOfInterest}
    \url{http://developers.sensorup.com/docs/#featureOfInterest_post}
\end{minipage}

Attribute:
\begin{itemize}
    \item \emph{public final String name} Dieser String repräsentiert den Namen des FeatureOfInterest in der Datenbank.
    \item \emph{public final String description} Dieser String stellt repräsentiert die Beschreibung des FeatureOfInterest in der Datenbank.
    \item \emph{public final MultiNavigationLink<Observation> observationsLink} \multiLinkDescription{Observation}
    \item \emph{public final Map<String, Object> features} Die in einem FeatureOfInterest gekapselten Informationen werden als Map von String und Object abgelegt.
    Jeder String identifiziert hierbei eindeutig ein Objekt.
\end{itemize}
Methoden:
\begin{itemize}
    \item \emph{public FeatureOfInterest(String id, String selfUrl, boolean relative, String description, String name, MultiNavigationLink<Observation> observationsLink, Map<String, Object> features)}
    \constructorDescription{FeatureOfInterest}
    \item \emph{public boolean equals(Object obj)} \equalsDescription{FeatureOfInterest}
\end{itemize}

%HistoricalLocation
%TODO: Fix class-table size
\rule{\textwidth}{0.4pt}
\class{HistoricalLocation extends Sensorthing<HistoricalLocation>}
\begin{minipage}{0.4\textwidth}
    \begin{figure}[H]
        {\centering\includegraphics[width=0.95\textwidth]{media/backend/modell/classes/HistoricalLocation.png}}
    \end{figure}
    \end{minipage} \hfill
\begin{minipage}{0.6\textwidth}
    \sensorthingsClassDescription{HistoricalLocation}{HistoricalLocations}
    \url{http://developers.sensorup.com/docs/#historicalLocations_get}
\end{minipage}

Attribute:
\begin{itemize}
    \item \emph{public final Instant time} Der zeitpunkt zudem die Ortsangabe gehört.
    \item \emph{public final SingleNavigationLink<Thing> thingLink} \singleLinkDescription{Thing}
    \item \emph{public final MultiNavigationLink<Location> locationsLink} \multiLinkDescription{Location}
\end{itemize}
Methoden:
\begin{itemize}
    \item \emph{public HistoricalLocation(String id, String selfUrl, boolean relative, Instant time, SingleNavigationLink<Thing> thingLink, MultiNavigationLink<Location> locationsLink)}
    \constructorDescription{HistoricalLocation}
    \item \emph{public boolean equals(Object obj)} \equalsDescription{HistoricalLocation}
\end{itemize}

%Location
%TODO: Fix class-table size
\rule{\textwidth}{0.4pt}
\class{Location extends Sensorthing<Location>}
\begin{minipage}{0.4\textwidth}
    \begin{figure}[H]
        {\centering\includegraphics[width=0.95\textwidth]{media/backend/modell/classes/Location.png}}
    \end{figure}
    \end{minipage} \hfill
\begin{minipage}{0.6\textwidth}
    \sensorthingsClassDescription{Location}{Locations}
    \url{http://developers.sensorup.com/docs/#locations_get}
\end{minipage}

Attribute:
\begin{itemize}
    \item \emph{public final String name} Dieser String repräsentiert den Namen der Location in der Datenbank.
    \item \emph{public final String description} Dieser String stellt repräsentiert die Beschreibung der Location in der Datenbank.
    \item \emph{public final Point location} Eine Ortsangabe in Form von Geo-Koordinaten.
    \item \emph{public final MultiNavigationLink<HistoricalLocation> historicalLocationsLink} \multiLinkDescription{HistoricalLocation}
    \item \emph{public final MultiNavigationLink<Thing> thingsLink} \multiLinkDescription{Thing}
\end{itemize}
Methoden:
\begin{itemize}
    \item \emph{public Location(String id, String selfUrl, boolean relative, String name, String description, Point location, MultiNavigationLink<HistoricalLocation> historicalLocationsLink, MultiNavigationLink<Thing> thingsLink)}
    \constructorDescription{Location}
    \item \emph{public boolean equals(Object obj)} \equalsDescription{Location}
\end{itemize}

%Observation
%TODO: Fix class-table size
\rule{\textwidth}{0.4pt}
\class{Observation extends Sensorthing<Observation> implements SensorthingsTimeStamp}
\begin{minipage}{0.4\textwidth}
    \begin{figure}[H]
        {\centering\includegraphics[width=0.95\textwidth]{media/backend/modell/classes/Observation.png}}
    \end{figure}
    \end{minipage} \hfill
\begin{minipage}{0.6\textwidth}
    \sensorthingsClassDescription{Observation}{Observations}
    \url{http://developers.sensorup.com/docs/#observations_post}
\end{minipage}

Attribute:
\begin{itemize}
    \item \emph{public final Instant phenomenonTime} Test %TODO
    \item \emph{public final Double result} Der Wert den die Beobachtung annimmt
    \item \emph{public final Instant resultTime} Test %TODO
    \item \emph{public final SingleNavigationLink<Datastream> datastreamLink} \singleLinkDescription{Datastream}
    \item \emph{public final SingleNavigationLink<FeatureOfInterest> featureOfInterestLink} \singleLinkDescription{FeatureOfInterest}
\end{itemize}
Methoden:
\begin{itemize}
    \item \emph{public Location(String id, String selfUrl, boolean relative, String name, String description, Point location, MultiNavigationLink<HistoricalLocation> historicalLocationsLink, MultiNavigationLink<Thing> thingsLink)}
    \constructorDescription{Location}
    \item \emph{public boolean equals(Object obj)} \equalsDescription{Location}
\end{itemize}

\clearpage %opt
%PointDatum
\rule{\textwidth}{0.4pt}
\class{PointDatum}
public class PointDatum
\\\\
\begin{minipage}{0.25\textwidth}
    \begin{figure}[H]
        {\centering\includegraphics[width=0.95\textwidth]{media/backend/modell/classes/PointDatum.png}}
    \end{figure}
    \end{minipage} \hfill
\begin{minipage}{0.75\textwidth}
    Die Klasse PointDatum beschreibt einen spezifischen Datenpunkt an einem spezifischen Ort.
\end{minipage}

Attribute:
\begin{itemize}
    \item \emph{public final Point location} Ein Ort gegeben durch seine Geo-Koordinaten.
    \item \emph{public final double datum} Der Datenpunkt als numerischer Wert.
\end{itemize}
Methoden:
\begin{itemize}
    \item \emph{public PointDatum(Point location, double datum)}
    \constructorDescription{PointDatum}
\end{itemize}

%RestConstants
\rule{\textwidth}{0.4pt}
\class{RestConstants}
public final class RestConstants
\\\\
\begin{minipage}{0.4\textwidth}
    \begin{figure}[H]
        {\centering\includegraphics[width=0.95\textwidth]{media/backend/modell/classes/RestConstants.png}}
    \end{figure}
    \end{minipage} \hfill
\begin{minipage}{0.6\textwidth}
    Die Klasse RestConstants kapselt Konstanten, die bei der Kommunikation mit der Rest-API benötigt werden.
    Die Klasse ist final deklariert, damit von ihr nicht geerbt werden kann.
    In Kombination mit dem privaten Konstruktor sorgt dies dafür, dass die Klasse als reine Container-Klasse weder instanziiert noch geerbt werden kann.
\end{minipage}

Attribute:
\begin{itemize}
    \item \emph{public static final String ENTRY\_POINT} Eine URL, die einen Einstigespunkt in die \gls{SensorThings API} beschreibt.
    In einem static Block wird diese Konstante aus der Datei rest\_constants.properties gelesen
\end{itemize}
Methoden:
\begin{itemize}
    \item \emph{private RestConstants()}
    \constructorDescription{RestConstants}
    Der private Konstruktor bewirkt das die Klasse von außen nicht instanziiert werden kann
\end{itemize}

\subsubsection{Controller}
%NavigationLink
\class{NavigationLink}
public abstract class NavigationLink<SensorthingT extends Sensorthing<SensorthingT>>
\\\\
\begin{minipage}{0.3\textwidth}
    \begin{figure}[H]
        {\centering\includegraphics[width=0.95\textwidth]{media/backend/controller/classes/NavigationLink.png}}
    \end{figure}
    \end{minipage} \hfill
\begin{minipage}{0.7\textwidth}
    Die abstrakte Klasse NavigationLink beschreibt einen Verweis auf ein Objekt aus dem Model der \gls{SensorThings API}.
    Die Klasse nimmt als Generic den Typ des Objektes auf welchen der Verweis zeigt.
\end{minipage}

Attribute:
\begin{itemize}
    \item \emph{public final String url} Die URL, welche auf das verwiesene Element in der Datenbank zeigt.
\end{itemize}
Methoden:
\begin{itemize}
    \item \emph{public NavigationLink(String url, boolean relative)}
    \constructorDescription{NavigationLink}
    \relativeDescription
    \item \emph{protected JSONObject getJson()} Die Methode Holt das unter dem URL abgelegte JSONObject aus der Datenbank.
    Wird kein solches Objekt unter dem angegebene Link in der Datenbank gefunden wird null zurück gegeben.
\end{itemize}

%SingleNavigationLink
\rule{\textwidth}{0.4pt}
\class{SingleNavigationLink}
public abstract class SingleNavigationLink<SensorthingT extends Sensorthing<SensorthingT>> extends NavigationLink<SensorthingT>
\\\\
\begin{minipage}{0.55\textwidth}
    \begin{figure}[H]
        {\centering\includegraphics[width=0.95\textwidth]{media/backend/controller/classes/SingleNavigationLink.png}}
    \end{figure}
    \end{minipage} \hfill
\begin{minipage}{0.45\textwidth}
    Die abstrakte Klasse SingleNavigationLink stellt einen Verweis auf ein einzelnes Object in der \gls{SensorThings API} da.
\end{minipage}

Methoden:
\begin{itemize}
    \item \emph{public SingleNavigationLink(String url, boolean relative)}
    \constructorDescription{SingleNavigationLink}
    \relativeDescription
    \item \emph{public abstract Sensorthing<SensorthingT> get(SensorthingController<SensorthingT> controller)}
    Die Funktion gibt das durch die URL repräsentierte Objekt aus der \gls{SensorThings API} zurück.
    Der Methode muss eine passender Controller für das gewollte Element mit übergeben werden.
\end{itemize}

%SingleLocalLink
\rule{\textwidth}{0.4pt}
\class{SingleLocalLink}
public class SingleLocalLink<SensorthingT extends Sensorthing<SensorthingT>> extends SingleNavigationLink<SensorthingT>
\\\\
\begin{minipage}{0.5\textwidth}
    \begin{figure}[H]
        {\centering\includegraphics[width=0.95\textwidth]{media/backend/controller/classes/SingleLocalLink.png}}
    \end{figure}
    \end{minipage} \hfill
\begin{minipage}{0.5\textwidth}
    Die Klasse SingleLocalLink stellt einen Verweis auf ein Object aus einer Datenbank der \gls{SensorThings API} da.
    In diesem Fall ist das Object auf welches Verweisen wird zwar unter einem angegebenen Link in der Datenbank verfügbar, jedoch für die primäre Nutzung bereits Local als Objekt gecached.
\end{minipage}

Attribute:
\begin{itemize}
    \item \emph{public final Sensorthing<SensorthingT> cachedSensorthing} Das bereits local gespeicherte Object aus der \gls{SensorThings API}.
\end{itemize}
Methoden:
\begin{itemize}
    \item \emph{public SingleLocalLink(String url, boolean relative, Sensorthing<SensorthingT> cachedSensorthing)}
    \constructorDescription{SingleLocalLink}
    \relativeDescription
    \item \emph{public Sensorthing<SensorthingT> get(SensorthingController<SensorthingT> controller)}
    Die Methode implementiert die abstrakte Methode aus der abstrakten Klasse SingleNavigationLink. In dieser Implementierung wird das bereits local gespeicherte Objekt zurück gegeben.
\end{itemize}

%SingleOnlineLink
\rule{\textwidth}{0.4pt}
\class{SingleOnlineLink}
public class SingleOnlineLink<SensorthingT extends Sensorthing<SensorthingT>> extends SingleNavigationLink<SensorthingT>
\\\\
\begin{minipage}{0.5\textwidth}
    \begin{figure}[H]
        {\centering\includegraphics[width=0.95\textwidth]{media/backend/controller/classes/SingleOnlineLink.png}}
    \end{figure}
    \end{minipage} \hfill
\begin{minipage}{0.5\textwidth}
    Die Klasse SingleLocalLink stellt einen Verweis auf ein Object aus einer Datenbank der \gls{SensorThings API} da.
    In diesem Fall ist das Object bei der Initialisierung lediglich in der Datenbank der \gls{SensorThings API} gegeben.
    Beim ersten Aufruf des Objektes in dieser Klasse wird das Objekt aus der Datenbank geladen und ein lokaler Link auf das Objekt erstellt.
\end{minipage}

Attribute:
\begin{itemize}
    \item \emph{private SingleLocalLink<SensorthingT> cache} Sobald das Objekt das erste Mal aus der Datenbank gelesen wurde wird es in diesem Attribut als SingleLocalLink gecached.
\end{itemize}
Methoden:
\begin{itemize}
    \item \emph{public SingleOnlineLink(String url, boolean relative)}
    \constructorDescription{SingleOnlineLink}
    \relativeDescription
    \item \emph{public Sensorthing<SensorthingT> get(SensorthingController<SensorthingT> controller)}
    Beim ersten Aufruf wird das Objekt aus der Datenbank geladen und zurück gegeben.
    Dabei wird zusätzlich die Rückgabe der Datenbank gecached.
    Bei allen weiteren Aufrufen wird dann das lokal gespeicherte Element zurück gegeben.
\end{itemize}

%MultiNavigationLink
\rule{\textwidth}{0.4pt}
\class{MultiNavigationLink}
public abstract class MultiNavigationLink<SensorthingT extends Sensorthing<SensorthingT>> extends NavigationLink<SensorthingT>
\\\\
\begin{minipage}{0.5\textwidth}
    \begin{figure}[H]
        {\centering\includegraphics[width=0.95\textwidth]{media/backend/controller/classes/MultiNavigationLink.png}}
    \end{figure}
    \end{minipage} \hfill
\begin{minipage}{0.5\textwidth}
    Die abstrakte Klasse MultiNavigationLink stellt einen Verweis auf eine Gruppe von Objekten in der \gls{SensorThings API} da.
\end{minipage}

Methoden:
\begin{itemize}
    \item \emph{public MultiNavigationLink(String url, boolean relative)}
    \constructorDescription{MultiNavigationLink}
    \relativeDescription
    \item \emph{public abstract ArrayList<SensorthingT> get(SensorthingController<SensorthingT> controller)}
    Gibt alle Objekte zurück, die unter der gegebenen URL in der Datenbank der \gls{SensorThings API} zur Verfügung stehen.
\end{itemize}

%MultiLocalLink
\rule{\textwidth}{0.4pt}
\class{MultiLocalLink}
public class MultiLocalLink<SensorthingT extends Sensorthing<SensorthingT>> extends MultiNavigationLink<SensorthingT>
\\\\
\begin{minipage}{0.5\textwidth}
    \begin{figure}[H]
        {\centering\includegraphics[width=0.95\textwidth]{media/backend/controller/classes/MultiLocalLink.png}}
    \end{figure}
    \end{minipage} \hfill
\begin{minipage}{0.5\textwidth}
    Die Klasse MultiLocalLink stellt einen Verweis auf mehrere Objekte aus einer Datenbank der \gls{SensorThings API} da.
    In diesem Fall sind die Objekt auf welche Verweisen wird zwar unter einem angegebenen Link in der Datenbank verfügbar, jedoch für die primäre Nutzung bereits Local als Objekt gecached.
\end{minipage}

Methoden:
\begin{itemize}
    \item \emph{public MultiLocalLink(String url, boolean relative, ArrayList<SensorthingT> cachedSensorthing)}
    \constructorDescription{MultiLocalLink}
    \relativeDescription
    \item \emph{public ArrayList<SensorthingT> get(SensorthingController<SensorthingT> controller)}
    Gibt alle Objekte zurück, die unter der gegebenen URL in der Datenbank der \gls{SensorThings API} zur verfügung stehen.
    In diesem Fall werden die in der Instanz gespeicherten Objekte als ArrayList zurück gegeben.
\end{itemize}

%MultiOnlineLink
\rule{\textwidth}{0.4pt}
\class{MultiOnlineLink}
public class MultiOnlineLink<SensorthingT extends Sensorthing<SensorthingT>> extends MultiNavigationLink<SensorthingT>
\\\\
\begin{minipage}{0.5\textwidth}
    \begin{figure}[H]
        {\centering\includegraphics[width=0.95\textwidth]{media/backend/controller/classes/MultiOnlineLink.png}}
    \end{figure}
    \end{minipage} \hfill
\begin{minipage}{0.5\textwidth}
    Die Klasse MultiOnlineLink stellt einen Verweis auf mehrere Objekte aus einer Datenbank der \gls{SensorThings API} da.
    In diesem Fall sind die Objekte bei der Initialisierung lediglich in der Datenbank der \gls{SensorThings API} gegeben.
    Beim ersten Aufruf der Objekte in dieser Klasse werden die Objekte aus der Datenbank geladen und ein lokaler Link auf die gruppe von Objekten erstellt.
\end{minipage}

Attribute:
\begin{itemize}
    \item \emph{private MultiLocalLink<SensorthingT> cache} Sobald die Objekte das erste Mal aus der Datenbank gelesen wurden werden diese in diesem Attribut als MultiLocalLink gecached.
\end{itemize}
Methoden:
\begin{itemize}
    \item \emph{public MultiOnlineLink(String url, boolean relative)}
    \constructorDescription{MultiOnlineLink}
    \relativeDescription
    \item \emph{public ArrayList<SensorthingT> get(SensorthingController<SensorthingT> controller)}
    Gibt alle Objekte zurück, die unter der gegebenen URL in der Datenbank der \gls{SensorThings API} zur verfügung stehen.
    Beim ersten Aufruf werden die Objekte aus der Datenbank geladen und zurück gegeben.
    Dabei wird zusätzlich die Rückgabe der Datenbank gecached.
    Bei allen weiteren Aufrufen werden dann die lokal gespeicherte Element zurück gegeben.
\end{itemize}
%Interpolation-Classes
%Interpolation
\clearpage %opt
\rule{\textwidth}{0.4pt}
\class{Interpolation}
public abstract class Interpolation
\\\\
\begin{minipage}{0.5\textwidth}
    \begin{figure}[H]
        {\centering\includegraphics[width=0.95\textwidth]{media/backend/controller/classes/Interpolation.png}}
    \end{figure}
    \end{minipage} \hfill
\begin{minipage}{0.5\textwidth}
    Die abstrakte Klasse Interpolation stellt Interpolationsalgorithmen da, mit welchen die Sensordaten aus der Datenbank der \gls{SensorThings API} interpoliert werden können.
    Die Algorithmen interpolieren die Messpunkte nicht durchgehend sondern nur an gleichmäßig verteilten Punkten.
\end{minipage}

Methoden:
\begin{itemize}
    \item \emph{public Interpolation()}
    \constructorDescription{Interpolation}
    \item \emph{public PointDatum[] interpolate(Envelope envelope, Instant time, TemporalAmount range, ObservedProperty observedProperty)}
    Die Methode interpoliert alle Daten zu dem angegebenen ObservedProperty in einem beschränkten Bereich zu einem festem Zeitpunkt.
    Zur Angabe des Zeitpunktes findet als Instanz der Java-eigenen Klasse java.time.Instant statt.
    Für die Angabe der Region wird die Klasse org.locationtech.jts.geom.Envelope verwendet.
    Diese Klasse Envelope stellt einen rechteckigen Bereich da.
    Das Attribut range gibt an, aus welchem zeitlichen Bereich die für die Interpolation berücksichtigten Messwerte stammen
    \item \emph{protected abstract PointDatum[] interpolateCoordinates(Envelope envelope, ArrayList<Coordinate> coordinates)}
    Die Methode interpoliert Daten die im Attribut coordinates gegben sind.
    Das Attribut coordinates besteht aus einer Liste von org.locationtech.jts.geom.Coordinate Instanzen.
    Eine Instanz besteht hierbei aus den beiden Geo-Koordinaten und einem Messwert an der definierten Stelle.
    Zudem wird eine org.locationtech.jts.geom.Envelope Instanz übergeben, die den zu interpolierenden Bereich angibt.
    Die Rückgabe besteht aus einem Array von PointDatum Instanzen, die die interpolierten Werte abbilden
\end{itemize}
\clearpage %opt
%DefaultInterpolation
\rule{\textwidth}{0.4pt}
\class{DefaultInterpolation}
public class DefaultInterpolation extends Interpolation
\\\\
%\begin{minipage}{0.5\textwidth}
    \begin{figure}[H]
        {\centering\includegraphics[width=0.7\textwidth]{media/backend/controller/classes/DefaultInterpolation.png}}
    \end{figure}
%\end{minipage} \hfill
%\begin{minipage}{0.5\textwidth}
    Die Klasse DefaultInterpolation implementiert den Standard-Interpolationsalgorithmus.
    Dieser Algorithmus verwendet hierfür Barnes Interpolation (siehe \url{https://en.wikipedia.org/wiki/Barnes_interpolation}).
    Diese Art der Interpolation kann besonders gut mit unregelmäßig verteilten Messpunkten umgehen.
    BarnesInterpolation wird speziell für die Interpolation von Wetterdaten verwendet, weswegen er für die hier interpolierten Werte sehr gut geignet ist.
    Für die Interpolation wird auf die Bibliothek org.geotools \url{https://geotools.org/} zurückgegriffen.
    Die BarnesInterpolation ist hier bereits als Klasse implementiert (\href{http://docs.geotools.org/latest/javadocs/org/geotools/process/vector/BarnesSurfaceInterpolator.html}{siehe}).
%\end{minipage}
\\
Attribute:
\begin{itemize}
    \item \emph{public static final String MAPPING} \mappingDescription
\end{itemize}
Methoden:
\begin{itemize}
    \item \emph{public DefaultInterpolation()}
    \constructorDescription{DefaultInterpolation}
    \item \emph{public PointDatum[] interpolate(Envelope envelope, Instant time, TemporalAmount range, ObservedProperty observedProperty)}
    Von der Klasse Interpolation geerbte Methode. Die Methode wird hier um ein Mapping auf die Webschnittstelle der Spring-Boot-Anwendung ergänzt
   \item \emph{protected PointDatum[] interpolateCoordinates(Envelope envelope, ArrayList<Coordinate> coordinates)}
    Diese Methode wird von der Klasse Interpolation geerbt und hier mit der BarnesInterpolation implementiert
\end{itemize}

%Controller for SensorThings
%SensorthingController
\rule{\textwidth}{0.4pt}
\class{SensorthingController}
public abstract class SensorthingController<SensorthingT extends Sensorthing<SensorthingT>>
\\\\
\begin{minipage}{0.4\textwidth}
    \begin{figure}[H]
        {\centering\includegraphics[width=0.95\textwidth]{media/backend/controller/classes/SensorthingsController.png}}
    \end{figure}
    \end{minipage} \hfill
\begin{minipage}{0.6\textwidth}
    Die abstrakte Klasse SensorthingsController kapselt die Funktionen der Controller für Objekte die von Sensorthings erben.
\end{minipage}

Methoden:
\begin{itemize}
    \item \emph{public SensorthingController()}
    \constructorDescription{SensorthingController}
    \item \emph{public abstract ArrayList<SensorthingT> getAll()}
    Gibt alle Objekte entsprechenden Typs zurück, die in der Datenbank enthalten sind.
    \item \emph{public ArrayList<SensorthingT> get(MultiNavigationLink<SensorthingT> navigationLink)}
    Gibt alle Sensorthings zurück, die durch den MultiNavigationLink gegeben sind.
    \item \emph{public Sensorthing<SensorthingT> get(SingleNavigationLink<SensorthingT> navigationLink)}
    Gibt das Sensorthing zurück, das durch den SingleNavigationLink gegeben ist.
    \item \emph{public abstract SensorthingT get(String id)}
    Gibt das Sensorthing zurück, das durch unter der gegebenen id in der Datenbank abgelegt ist.
    \item \emph{public ArrayList<SensorthingT> multiBuild(JSONObject json)}
    Baut alle in dem gegebenen JSONObject enthaltenen Sensorthings und gibt diese zurück.
    Falls das JSONObject nicht zu dem entsprechenden Sensorthing umgewandelt werden kann, wird null zurück gegeben.
    \item \emph{public abstract SensorthingT singleBuild(JSONObject json)}
    Baut das als JSONObject übergebene Sensorthing
    Falls das JSONObject nicht zu dem entsprechenden Sensorthing umgewandelt werden kann, wird null zurück gegeben.
\end{itemize}

%DatastreamController
\rule{\textwidth}{0.4pt}
\class{DatastreamController}
public class DatastreamController extends SensorthingController<Datastream>
\\\\
\begin{minipage}{0.4\textwidth}
    \begin{figure}[H]
        {\centering\includegraphics[width=0.95\textwidth]{media/backend/controller/classes/DatastreamController.png}}
    \end{figure}
    \end{minipage} \hfill
\begin{minipage}{0.6\textwidth}
    \controllerDescription{DatastreamController}{Datastream}
\end{minipage}

Attribute:
\begin{itemize}
    \item \emph{public static final String MAPPING} \mappingDescription
\end{itemize}
Methoden:
\begin{itemize}
    \item \emph{public DatastreamController()}
    \constructorDescription{DatastreamController}
    \item \emph{public ArrayList<Datastream> getAll()}
    \extendsSensorthingController
    \item \emph{public ArrayList<Datastream> get(Thing thing)}
    Gibt alle Datastream Objekte zurück, die im gegebenen Thing enthalten sind.
    \item \emph{public ArrayList<Datastream> get(Sensor sensor)}
    Gibt alle Datastream Objekte zurück, die im gegebenen Sensor enthalten sind.
    \item \emph{public Datastream get(Thing thing, ObservedProperty observedProperty)}
    Gibt alle Datastream Objekte zurück, die vom gegebenen Thing mit dem gegebenen ObservedProperty existieren.
    \item \emph{public Datastream get(Sensor sensor, ObservedProperty observedProperty)}
    Gibt alle Datastream Objekte zurück, die vom gegebenen Sensor mit dem gegebenen ObservedProperty existieren.
    \item \emph{public Datastream get(String id)}
    \extendsSensorthingController
    \item \emph{public Datastream singleBuild(JSONObject json)}
    \extendsSensorthingController
\end{itemize}

%FeatureOfInterestController
\rule{\textwidth}{0.4pt}
\class{FeatureOfInterestController}
public class FeatureOfInterestController extends SensorthingController<FeatureOfInterest>
\\\\
\begin{minipage}{0.4\textwidth}
    \begin{figure}[H]
        {\centering\includegraphics[width=0.95\textwidth]{media/backend/controller/classes/FeatureOfInterest.png}}
    \end{figure}
    \end{minipage} \hfill
\begin{minipage}{0.6\textwidth}
    \controllerDescription{FeatureOfInterestController}{FeatureOfInterest}
\end{minipage}

Attribute:
\begin{itemize}
    \item \emph{public static final String MAPPING} \mappingDescription
\end{itemize}
Methoden:
\begin{itemize}
    \item \emph{public FeatureOfInterestController()}
    \constructorDescription{FeatureOfInterestController}
    \item \emph{public ArrayList<FeatureOfInterest> getAll()}
    \extendsSensorthingController
    \item \emph{public FeatureOfInterest get(String id)}
    \extendsSensorthingController
    \item \emph{public FeatureOfInterest singleBuild(JSONObject json)}
    \extendsSensorthingController
    \item \emph{public Point getLocationPoint(FeatureOfInterest foi)}
    Gibt einen Ort zurück, der als feature im gegebene FeatureOfInterest enthalten ist.
    Falls keine Ortsangabe existiert wird null zurück gegeben.
\end{itemize}

%HistoricalLocationController
\rule{\textwidth}{0.4pt}
\class{HistoricalLocationController}
public class HistoricalLocationController extends SensorthingController<HistoricalLocation>
\\\\
\begin{minipage}{0.4\textwidth}
    \begin{figure}[H]
        {\centering\includegraphics[width=0.95\textwidth]{media/backend/controller/classes/HistoricalLocationController.png}}
    \end{figure}
    \end{minipage} \hfill
\begin{minipage}{0.6\textwidth}
    \controllerDescription{HistoricalLocationController}{HistoricalLocation}
\end{minipage}

Attribute:
\begin{itemize}
    \item \emph{public static final String MAPPING} \mappingDescription
\end{itemize}
Methoden:
\begin{itemize}
    \item \emph{public HistoricalLocationController()}
    \constructorDescription{HistoricalLocationController}
    \item \emph{public ArrayList<HistoricalLocation> getAll()}
    \extendsSensorthingController
    \item \emph{public HistoricalLocation get(String id)}
    \extendsSensorthingController
    \item \emph{public HistoricalLocation singleBuild(JSONObject json)}
    \extendsSensorthingController
\end{itemize}

%LocationController
\rule{\textwidth}{0.4pt}
\class{LocationController}
public class LocationController extends SensorthingController<Location>
\\\\
\begin{minipage}{0.4\textwidth}
    \begin{figure}[H]
        {\centering\includegraphics[width=0.95\textwidth]{media/backend/controller/classes/LocationController.png}}
    \end{figure}
    \end{minipage} \hfill
\begin{minipage}{0.6\textwidth}
    \controllerDescription{LocationController}{Location}
\end{minipage}

Attribute:
\begin{itemize}
    \item \emph{public static final String MAPPING} \mappingDescription
\end{itemize}
Methoden:
\begin{itemize}
    \item \emph{public LocationController()}
    \constructorDescription{LocationController}
    \item \emph{public ArrayList<Location> getAll()}
    \extendsSensorthingController
    \item \emph{public Location get(String id)}
    \extendsSensorthingController
    \item \emph{public Location singleBuild(JSONObject json)}
    \extendsSensorthingController
\end{itemize}

%ObservationController
\rule{\textwidth}{0.4pt}
\class{ObservationController}
public class ObservationController extends SensorthingController<Observation>
\\\\
\begin{minipage}{0.4\textwidth}
    \begin{figure}[H]
        {\centering\includegraphics[width=0.95\textwidth]{media/backend/controller/classes/ObservationController.png}}
    \end{figure}
    \end{minipage} \hfill
\begin{minipage}{0.6\textwidth}
    \controllerDescription{ObservationController}{Observation}
\end{minipage}

Attribute:
\begin{itemize}
    \item \emph{public static final String MAPPING} \mappingDescription
\end{itemize}
Methoden:
\begin{itemize}
    \item \emph{public ObservationController()}
    \constructorDescription{ObservationController}
    \item \emph{public ArrayList<Location> getAll()}
    \extendsSensorthingController
    \item \emph{public Location get(String id)}
    \extendsSensorthingController
    \item \emph{public Location singleBuild(JSONObject json)}
    \extendsSensorthingController
    \item \emph{public ArrayList<Observation> get(Datastream datastream)}
    Gibt alle Observations eines Datastreams zurück
    \item \emph{public ArrayList<Observation> get(Envelope envelope, Instant time, TemporalAmount range, ObservedProperty observedProperty)}
    Gibt alle Observations in einem bestimmten zeitlichen Bereich zu und einem begrenzenten geographischen Bereich zurück, die ein bestimmtes ObservedProperty besitzen.
    Die zeitliche Abgrenzung findet über die beiden Parameter time und range statt. Hierbei ist time der Zeitpunkt und range ein zeitlicher Spielraum, in welchem die Datastreams um time liegen müssen.
\end{itemize}

%ObservedPropertyController
\rule{\textwidth}{0.4pt}
\class{ObservedPropertyController}
public class ObservedPropertyController extends SensorthingController<ObservedProperty>
\\\\
\begin{minipage}{0.4\textwidth}
    \begin{figure}[H]
        {\centering\includegraphics[width=0.95\textwidth]{media/backend/controller/classes/ObservedPropertyController.png}}
    \end{figure}
    \end{minipage} \hfill
\begin{minipage}{0.6\textwidth}
    \controllerDescription{ObservedPropertyController}{ObservedProperty}
\end{minipage}

Attribute:
\begin{itemize}
    \item \emph{public static final String MAPPING} \mappingDescription
\end{itemize}
Methoden:
\begin{itemize}
    \item \emph{public ObservedPropertyController()}
    \constructorDescription{ObservedPropertyController}
    \item \emph{public ArrayList<ObservedProperty> getAll()}
    \extendsSensorthingController
    \item \emph{public ObservedProperty get(String id)}
    \extendsSensorthingController
    \item \emph{public ObservedProperty singleBuild(JSONObject json)}
    \extendsSensorthingController
    \item \emph{public ObservedProperty get(Datastream datastream)}
    Gibt alle das ObservedProperty des gegebenen Datastreams zurück.
\end{itemize}

%SensorController
\rule{\textwidth}{0.4pt}
\class{SensorController}
public class SensorController extends SensorthingController<Sensor>
\\\\
\begin{minipage}{0.4\textwidth}
    \begin{figure}[H]
        {\centering\includegraphics[width=0.95\textwidth]{media/backend/controller/classes/SensorController.png}}
    \end{figure}
    \end{minipage} \hfill
\begin{minipage}{0.6\textwidth}
    \controllerDescription{SensorController}{Sensor}
\end{minipage}

Attribute:
\begin{itemize}
    \item \emph{public static final String MAPPING} \mappingDescription
\end{itemize}
Methoden:
\begin{itemize}
    \item \emph{public SensorController()}
    \constructorDescription{SensorController}
    \item \emph{public ArrayList<Sensor> getAll()}
    \extendsSensorthingController
    \item \emph{public Sensor get(String id)}
    \extendsSensorthingController
    \item \emph{public Sensor singleBuild(JSONObject json)}
    \extendsSensorthingController
    \item \emph{public ArrayList<Sensor> get(Thing thing)}
    Gibt alle das Sensoren zum gegebenen Thing zurück.
    \item \emph{public Sensor get(Datastream datastream)}
    Gibt den Sensor zum gegebenen Datastream zurück.
\end{itemize}

%ThingController
\rule{\textwidth}{0.4pt}
\class{ThingController}
public class ThingController extends SensorthingController<Thing>
\\\\
\begin{minipage}{0.4\textwidth}
    \begin{figure}[H]
        {\centering\includegraphics[width=0.95\textwidth]{media/backend/controller/classes/ThingController.png}}
    \end{figure}
    \end{minipage} \hfill
\begin{minipage}{0.6\textwidth}
    \controllerDescription{ThingController}{Thing}
\end{minipage}

Attribute:
\begin{itemize}
    \item \emph{public static final String MAPPING} \mappingDescription
\end{itemize}
Methoden:
\begin{itemize}
    \item \emph{public ThingController()}
    \constructorDescription{ThingController}
    \item \emph{public ArrayList<Thing> getAll()}
    \extendsSensorthingController
    \item \emph{public Thing get(String id)}
    \extendsSensorthingController
    \item \emph{public Thing singleBuild(JSONObject json)}
    \extendsSensorthingController
    \item \emph{public ArrayList<Thing> get(Envelope envelope)}
    Gibt alle Things in einem eingeschränkten geographischen Bereich da.
\end{itemize}

%UtilityController
\rule{\textwidth}{0.4pt}
\class{UtilityController}
public final class UtilityController
\\\\
\begin{minipage}{0.4\textwidth}
    \begin{figure}[H]
        {\centering\includegraphics[width=0.95\textwidth]{media/backend/controller/classes/UtilityController.png}}
    \end{figure}
    \end{minipage} \hfill
\begin{minipage}{0.6\textwidth}
    Die Klasse UtilityController kapselt nützliche Funktionen die von allen Controllern benutzt werden.
\end{minipage}

Methoden:
\begin{itemize}
    \item \emph{public UtilityController()}
    \constructorDescription{UtilityController}
    Der private Konstruktor führt dazu das die Klasse nicht instanziiert werden kann.
    \item \emph{public static UnitOfMeasurement buildUnitOfMeasurement(JSONObject json)}
    Diese Methode baut eine Maßeinheit aus einem JSONObject. Falls das JSONObject nicht als UnitOfMeasurement interpretiert werden kann wird null zurück gegeben.
    \item \emph{public static Instant buildTime(JSONObject json, String key)}
    Diese Methode baut einen Zeitpunkt als Instant aus dem Inhalt des JSONObject. Der Zeitpunkt muss unter dem key im JSONObject gefunden werden können.
    Wenn keine Instant erstellt werden kann wird null zurück gegeben.
    \item \emph{public static Point buildLocationPoint(JSONObject json)}
    Die Methode baut einen Punkt aus dem gegebenen JSONObject.
    \item \emph{public static Map<String, Object> buildProperties(JSONObject json)}
    Die Methode baut die im JSONObject gegebenen properties.
    \item \emph{public static JSONObject removeArrayWrapper(JSONObject json)}
    Die Methode entfernt unnötige Verschachtelungen im JSONObject 
\end{itemize}
