\tikzstyle{every node}=[draw=black,thick,anchor=west]
\tikzstyle{selected}=[draw=red,fill=red!30]
\tikzstyle{optional}=[dashed,fill=gray!50]

%Entwurfdaten
\section{Entwurfsdaten}
In diesem Kapitel werden die von \softwarename benötigten Daten aufgeführt.

\subsection{Ressourcen-Verzeichnis}
Da es sich bei \softwarename um eine Webanwendung handelt, werden im Folgenden die Ressourcen in Backend und Frontend aufgeschlüsselt.
Ressourcen, die nur zu Testzwecken benötigt werden, sind dabei die Kinder einer rot hinterlegten Test-Directory. Wird die Ressource für erweiterte Funktionen benötigt, so ist sie grau hinterlegt.

\subsubsection{Backend Ressourcen}
\begin{tikzpicture}[%
                grow via three points={one child at (0.5,-0.7) and
                                two children at (0.5,-0.7) and (0.5,-1.4)},
                edge from parent path={(\tikzparentnode.south) |- (\tikzchildnode.west)}]
        \node {main}
        child { node {resources}
                        child { node {rest\_constants.properties}}
                };
\end{tikzpicture}

\begin{tikzpicture}[%
                grow via three points={one child at (0.5,-0.7) and
                                two children at (0.5,-0.7) and (0.5,-1.4)},
                edge from parent path={(\tikzparentnode.south) |- (\tikzchildnode.west)}]
        \node [selected] {test}
        child { node {resources}
                        child { node {alive\_datastream.json}}
                        child { node {alive\_featureofinterest.json}}
                        child { node {alive\_historicallocation.json}}
                        child { node {alive\_location.json}}
                        child { node {alive\_observation.json}}
                        child { node {alive\_observedproperty.json}}
                        child { node {alive\_sensor.json}}
                        child { node {alive\_thing.json}}
                };
\end{tikzpicture}

\subsubsection{Frontend Ressourcen}
\begin{tikzpicture}[%
                grow via three points={one child at (0.5,-0.7) and
                                two children at (0.5,-0.7) and (0.5,-1.4)},
                edge from parent path={(\tikzparentnode.south) |- (\tikzchildnode.west)}]
        \node {main}
        child { node {resources}
                        child { node {language}
                                        child { node {de\_DE.properties}}
                                        child { node [optional] {en\_EN.properties}}
                                }
                        child [missing] {}
                        child [missing] {}
                        child { node {icon}
                                        child { node {flag}
                                                        child { node {de\_DE\_flag.svg}}
                                                        child { node [optional] {en\_EN\_flag.svg}}
                                                }
                                        child [missing] {}
                                        child [missing] {}
                                        child { node {\softwarename\_logo.svg}}
                                }
                };
\end{tikzpicture}

\subsection{Dateneinträge}
Im Frontend werden clientseitig lediglich die \glspl{Cookie} in Form von Dateien auf dem Computer gespeichert.
Die folgenden Cookies werden dabei benötigt:
\begin{itemize}
        \item \gls{CookieNotice}-\gls{Cookie}: Wurde die \gls{CookieNotice} akzeptiert?
        \item \glspl{Cookie} zur Speicherung des Kartenausschnitts bei Verlassen
        \item \glspl{Cookie} zur Speicherung des Betriebsmodus bei Verlassen
\end{itemize}
Das Backend legt hingegen keine Dateien an und verwendet ausschließlich die \gls{SensorThings API} zum Instanziieren der Models.
